\documentclass{article}
\usepackage{appendix}
\usepackage[utf8]{inputenc}
\usepackage{mathtools}
\usepackage[affil-it]{authblk}
\usepackage[round]{natbib}
\usepackage{newtxtext,newtxmath} % Time New Rome 
\usepackage{enumerate}
\usepackage{caption} 
\usepackage{graphicx}
\usepackage{booktabs, multirow}
%\includegraphics{pdf/image}
\usepackage{siunitx}                        % added
\usepackage{threeparttable}
\usepackage{tabularx}
\usepackage{longtable}
%\usepackage{blindtext}
\usepackage{amsmath,amsfonts,amssymb}
\usepackage{mathtools}
\usepackage{dcolumn}
\usepackage{rotating}
\usepackage{float}
\usepackage{footmisc}
\let\oldfootnote\footnote
\renewcommand\footnote[1]{%
\oldfootnote{\hspace{1.5mm}#1}}
%\usepackage[colorlinks=]{hyperref}
\usepackage{hyperref}
\hypersetup{
    colorlinks=true,
%    allcolors= black,
    citecolor =blue,
    linkcolor=blue,
    filecolor=white,      
    urlcolor=black,
}


\setlength{\footnotemargin}{4mm}
\setlength{\footnotesep}{0.2cm}
\setlength{\skip\footins}{1cm}
\setlength{\parindent}{1em}
\renewcommand{\baselinestretch}{1.5}
%---------------------------------------------
\title{Algorithm and FOCs for PP}
\author{Draft}
\date{\today}

\begin{document}

\maketitle
\section{A tentative algorithm}
	To simply the notation, I do not use $i$ and $m$ indices to indicate individual and her type since they just represent different parameters in calculation. More than that, some notations in the paper are not very intuitive. I change them a little bit. \par 
	Preparation: Discretize observed state variables, $a$, $h$, and $z$ to $N$ points for each of them. Totally, there are $N^3$ possible states. Assume that $\Omega_t = \{ a_t, h_t, z_t \}$ is the state at period $t$ and $T+1$ is the threshold period of retirement. Backward induction solving this problem is as follows. \par
		\begin{enumerate}[{}1{).}]
			\item After retired, no labor supply but just consumption. Given a state at period $T+1$, the problem is deterministic without uncertainty. Therefore, the value function at $T+1$ is $V(\Omega_{T+1})$. More than that, $V(\Omega_{T+1})$ is just a function of $a_{T+1}$. So given a set of $a_{T+1}$, the value function $V(\Omega_{T+1})$ can be approximated by linear interpolation or GPR.
			
			\item At period $T$, based on the employment status and which industry the employee is, there are 10 choice-specific value functions. Let $j^k$ and $ k \in \{0, 1, 2, ..., 9\}$ be the discrete choice set. $j^0$ means unemployed and $j^k$ for $k>0$ means working for industry $k$. Let $W(\Omega_t, j^k)$ be choice specific value function at period $t$.
				
				\begin{enumerate} [{}a{).}]
					\item If unemployed at $T$, $W(\Omega_{T}, j^0) = \underset{c_T}{max} \{u(c_T, 0) + b + V(\Omega_{T+1})\}$, s.t. $c_T + a_{T+1} = a_T (1+r) +TR_T$. This problem is still deterministic. $W(\Omega_{T}, j^0)$ again is just a function of $a_t$. Based on endogenous grid method, it's easy to get the choice-specific policy function $c_T (\Omega_T, j^0) $.
					
					\item If employed at $T$ at industry $k (\geq 1)$, the employee is either in performance pay or not. Let $W^p(\Omega_{T}, j^k) $, where $p = \{0, 1\}$, be the value function of two different status, i.e. working under performance pay ($p=1$) or not ($p=0$).  It has 
						$$ W^p(\Omega_{T}, j^k) = \underset{c_T, l_T}{max} \Big\{u(c_T, l_T) + E(V(\Omega_{T+1}))\Big\} $$
					s.t.
						$$c_T + a_{T+1} = s_T l_T h_T (1 - \tau_T) + a_T(1+r) $$
						$$ h_{T+1} = l_T^{\gamma_l} h_T^{\gamma_h} + (1 - \delta_1)h_T$$
					The FOCs are:
						$$u_c(c_T, l_T) = E\Big(\frac{ \partial V(\Omega_{T+1})}{\partial a_{T+1}}\Big)$$
						$$u_l(c_T, l_T) = -E\Big(\frac{ \partial V(\Omega_{T+1})}{\partial a_{T+1}} * s_T h_T (1-\tau_T)\Big)$$
					After rearrangement, 
							$$c_T = \Big(E(\frac{\frac{ \partial V(\Omega_{T+1})}{\partial a_{T+1}}}{\chi_c})\Big)^{-1/\iota}$$
							$$l_T = \Big(\frac{E\big(\frac{ \partial V(\Omega_{T+1})}{\partial a_{T+1}} s_T h_T (1-\tau_T) \big)}{\chi_l}\Big)^{1/\psi}$$
					In this process, the expectation is taken with respect to the shock on income, which can be calculated using Gauss-Hermite rule. \par
					
					After that, the choice-specific value function is:
						$$ W(\Omega_{T}, j^k) = W^0(\Omega_{T}, j^k)(1 - Prob(p=1)) +  W^1(\Omega_{T}, j^k)Prob(p=1) $$
					where $Prob(p=1)$ is the probability that the employees work under performance pay. \par
					\textit{Potential algorithm to find optimal $c_T$ and $l_T$ :} \par
					For a given state at period $T$, $\Omega_T=\{a_T, h_T, \theta_T\}$, guess initial $c_0$ and $l_0$. Calculate $a_{t+1}$ for each quadrature point. Get the expectation and new $c$ and $l$. Repeat until it converges. \par
					Then policy function is $c^p(\Omega_T, j^k)$ and $l^p(\Omega_T, j^k)$. The value function $W^p(\Omega_{T}, j^k)$ can be calculated based on the policy functions. Then GPR can be used to approximate these functions. 
				\end{enumerate}  
			\item At period $t = 1, 2, ...., T-1$, assume that for $k \geq 1$, $c^p(\Omega_{t+1}, j^k)$, $l^p(\Omega_{t+1}, j^k)$ and $W^p(\Omega_{t+1}, j^k)$are known. Also, when unemployed, i.e. $k=0$, $c(\Omega_{t+1}, j^0)$ and $W(\Omega_{t+1}, j^0)$ are known.
				
				\begin{enumerate}[{}a{).}]
					\item If unemployed at $t$, then
						\begin{align*}
							\begin{split}
								W(\Omega_{t}, j^0) &=
									\begin{multlined}[t]
										\underset{c_t}{max}\Bigg\{u(c_t, 0) + b + \beta\bigg\{(1-\lambda_u)E\Big(W(\Omega_{t+1}, j^0) + \epsilon(j^0)\Big) \\ 
										+ \lambda_u E\Big[\underset{j'}{max}\big\{W(\Omega_{t+1}, j') + \epsilon(j') \big\}\Big] \bigg\} \Bigg\} 
									\end{multlined} \\
									&=
									\begin{multlined}[t]
										\underset{c_t}{max}\bigg\{u(c_t, 0) + \beta(1-\lambda_u) W(\Omega_{t+1}, j^0) + \beta \lambda_u ln\Big(\sum_{j'}\exp(W(\Omega_{t+1}, j'))\Big) \\ + b + \beta(1-\lambda_u)E(\epsilon(j^0)) + \Gamma \bigg\} 
									\end{multlined}
							\end{split}
						\end{align*}
					s.t. 
						$$c_t + a_{t+1} =  a_t(1+r)  + TR_t$$
					
					where $j' \in \{j^0, j^1, ..., j^9\}$. The FOC is:
						\[ u_c(c_t, 0) =  \beta(1-\lambda_u)\frac{\partial W(\Omega_{t+1}, j^0)}{\partial a_{t+1}} + \beta\lambda_u \sum_{j'}\frac{\exp(W(\Omega_{t+1}, j'))*\frac{\partial W(\Omega_{t+1}, j')}{\partial a_{t+1}}}{\sum_{j'}\exp(W(\Omega_{t+1}, j'))} \tag{1} \]
					
					By envelop condition, it has:
						\begin{align*}
							\begin{split}
								\frac{\partial W(\Omega_{t}, j^0)}{\partial a_{t}} &=\bigg( \beta(1-\lambda_u)\frac{\partial W(\Omega_{t+1}, j^0)}{\partial a_{t+1}} + \beta\lambda_u\sum_{j'}\frac{\exp W(\Omega_{t+1}, j')*\frac{\partial W(\Omega_{t+1}, j')}{\partial a_{t+1}}}{\sum_{j'}\exp(W(\Omega_{t+1}, j'))}\bigg) \frac{\partial a_{t+1}}{\partial a_{t}} \\
								& = (1+r)u_c(c_{t}, 0)
							\end{split}
						\end{align*}
						
					With the same spirit, we can get other $\frac{\partial W(\Omega_{t}, j^k)}{\partial a_{t}} = (1+r)u_c(c_t, 0)$ for $k =1, 2, .., 9$. Plug these derivatives in the FOC, it has:
						\begin{multline*}
							u_c(c_t, 0) = \beta(1+r)\Big((1-\lambda_u)u_c(c_{t+1}(\Omega_{t+1}, j^0), 0) \\
							+ \lambda_u \sum_{j'} P_{t+1}(j'|\Omega_{t+1}) u_c(c_{t+1}(\Omega_{t+1}, j'), 0)\Big) \tag{2}
						\end{multline*}
					where $P_{t+1}(j'|\Omega_{t+1}) = \frac{\exp(W(\Omega_{t+1}, j'))}{\sum_{j'}\exp(W(\Omega_{t+1}, j'))}$ and $\epsilon(j^k)$ is type specific shock, following extreme value distribution. $c_{t+1}(\Omega_{t+1}, j^0)$ is defined that
						$$ c_{t+1}(\Omega_{t+1}, j') = c^0_{t+1}(\Omega_{t+1}, j')(1 - Prob(p=1)) +  c^1_{t+1}(\Omega_{t+1}, j')Prob(p=1) $$ 
					
					There are two potential choices to get the policy function. Guess and iteration on equation (1) or endogenous grid method on equation (2). EGM is much more favorable.
					\item If employed at period $t$, then for $k \in \{1, 2,..., 9\}$ and different status of performance pay0, it has
						\begin{align*}
							\begin{split}
								W^p(\Omega_{t}, j^k) &= 
									\!\begin{multlined}[t] 
										\underset{c_t, l_t}{max}\Bigg\{u(c_t, l_t) + \beta\bigg( \lambda_l E\Big(W^p(\Omega_{t+1}, j^0) + \epsilon(j^0)\Big) \\
										+ \lambda_e E\big[\underset{j'}{max}\big\{W(\Omega_{t+1}, j') + \epsilon(j') \big\} \big] \\
										+ (1-\lambda_e - \lambda_l)E\Big(W(\Omega_{t+1}, j^k) + \epsilon(j^k)\Big) \bigg) \Bigg\} 
									\end{multlined} 
							\end{split}
						\end{align*}
					s.t. 
					$$c_t + a_{t+1} = s_t l_t h_t (1-\tau_t) + a_t(1+r) $$
					$$ h_{t+1} = f(h_t, l_t) =  l_t^{\gamma_l} h_t^{\gamma_h} + (1 - \delta_1)h_t$$
					where $j' \in \{j^1, j^2, .., j^9\}$. \\
					Following the same idea above and combing FOCs and envelop conditions,  it has:
						\begin{multline*}
							u_c(c_t, l_t) = \beta(1+r)\bigg[ \lambda_l E\Big(u_c\big(c_{t+1}(\Omega_{t+1}, j^0), l_t \big) \Big) \\
							 + \lambda_e E\Big( \sum_{j'}P_{t+1}(j'|\Omega_{t+1}) u_c\big(c_{t+1}(\Omega_{t+1}, j'), l_t \big) \Big) \\
							 + (1-\lambda_e-\lambda_l) E\Big( u_c\big(c^p_{t+1}(\Omega_{t+1}, j^k), l_t \big) \Big)  \bigg] \tag{3} 
						\end{multline*}	
						\begin{multline*}
							u_l(c_t, l_t) = -\beta\bigg[ \lambda_l E\Big(u_c\big(c_{t+1}(\Omega_{t+1}, j^0), l_t \big)\eta_t + J(j^0) \Big) \\
							+ \lambda_e E\Big( \sum_{j'}P_{t+1}(j'|\Omega_{t+1}) u_c\big(c_{t+1}(\Omega_{t+1}, j'), l_t \big)\eta_t + J(j') \Big) \\
							+ (1-\lambda_e-\lambda_l) E\Big( u_c\big(c^p_{t+1}(\Omega_{t+1}, j^k), l_t \big)\eta_t + J(j^k) \Big)  \bigg] \tag{4}
						\end{multline*}	
					where $\eta_t = (1+r)s_t h_t (1-\tau_t)$, $J(j^k) = \frac{\partial W(\Omega_{t+1}, j^k)}{\partial h_{t+1}} f'(h_t, l_t)$ for $k = \{0, 1,..., 9\}$,  $P_{t+1}(j'|\Omega_{t+1}) = \frac{\exp(W(\Omega_{t+1}, j'))}{\sum_{j'}\exp(W(\Omega_{t+1}, j'))}$ for $j' \in \{1,2,...,9\}$. And
						$$ c_{t+1}(\Omega_{t+1}, j') = c^0_{t+1}(\Omega_{t+1}, j')(1 - Prob(p=1)) +  c^1_{t+1}(\Omega_{t+1}, j')Prob(p=1) $$ 
				\end{enumerate}
		\end{enumerate}
\section{Questions}
	\begin{itemize}
		\item The biggest question is how to solve (3) and (4) in each period to find optimal consumption and labor supply. Brute force is too expensive to apply in Python. Guess and iteration is my current thought. But not sure how to prove it. 
		\item The range of labor supply $l$? [0, 1]? It is [0, 1].
		\item Should the status of performance pay be a state variable? It should be. Otherwise, it's impossible to determine the wage.
		\item The timing of the events in the model:
			\begin{enumerate}[{}a{)}]
				\item Receiving capital income (?).
				\item Making decisions about consumption and labor supply based on observed and unobserved state variables.
				\item Receiving income shock and labor income is realized. (capital income?)
				\item Randomly drawing a new status--employed or unemployed. If employed with a new offer, deciding which industry to work for.
				\item Randomly drawing whether getting performance pay.
			\end{enumerate}
		\item How to deal with taxes and transfers?
	\end{itemize}
%-----------------------------------------------------------------------------
\end{document}